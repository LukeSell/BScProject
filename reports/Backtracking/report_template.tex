\documentclass[]{final_report}
\usepackage{graphicx}
\usepackage{hyperref}


%%%%%%%%%%%%%%%%%%%%%%
%%% Input project details
\def\studentname{Luke Sell}
\def\reportyear{2019}
\def\projecttitle{Playing Games and Solving Puzzles Using AI}
\def\supervisorname{Iddo Tzameret}
\def\degree{BSc (Hons) in Computer Science}
\def\fullOrHalfUnit{Full Unit} % indicate if you are doing the project as a Full Unit or Half Unit
\def\finalOrInterim{Backtracking and Recursion} % indicate if this document is your Final Report or Interim Report

\begin{document}

\maketitle

%%%%%%%%%%%%%%%%%%%%%%
%%% Declaration

\chapter*{Declaration}

This report has been prepared on the basis of my own work. Where other published and unpublished source materials have been used, these have been acknowledged.

\vskip3em

Word Count: N/A

\vskip3em

Student Name: \studentname

\vskip3em

Date of Submission: 05/10/2019

\vskip3em

Signature: l.sell

\newpage

%%%%%%%%%%%%%%%%%%%%%%
%%% Table of Contents
\tableofcontents\pdfbookmark[0]{Table of Contents}{toc}\newpage

\chapter*{}
\addcontentsline{toc}{chapter}{}

Sudoku can be solved by backtracking, a variation of a depth first search, Sudoku has an initial and a goal test and actions must be performed by a solver to find the next state, so that the goal state may be found. A backtracking search on a Sudoku puzzle will search up to the depth of 81 if possible before trying other branches, however as the constraints of the puzzle can be checked during the search, not all states in a branch have to be checked. This results in an algorithm that is very efficient in both time and memory used, having at most 81 instances stored, and is complete, always finding a solution if there is one.

\label{endpage}



\end{document}

%\end{article}
