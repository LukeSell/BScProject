\documentclass[]{final_report}
\usepackage{graphicx}
\usepackage{hyperref}


%%%%%%%%%%%%%%%%%%%%%%
%%% Input project details
\def\studentname{Luke Sell}
\def\reportyear{2019}
\def\projecttitle{Playing Games and Solving Puzzles Using AI}
\def\supervisorname{Iddo Tzameret}
\def\degree{BSc (Hons) in Computer Science}
\def\fullOrHalfUnit{Full Unit} % indicate if you are doing the project as a Full Unit or Half Unit
\def\finalOrInterim{Techniques used by Human Solvers} % indicate if this document is your Final Report or Interim Report

\begin{document}

\maketitle

%%%%%%%%%%%%%%%%%%%%%%
%%% Declaration

\chapter*{Declaration}

This report has been prepared on the basis of my own work. Where other published and unpublished source materials have been used, these have been acknowledged.

\vskip3em

Word Count: N/A

\vskip3em

Student Name: \studentname

\vskip3em

Date of Submission: 05/10/2019

\vskip3em

Signature: l.sell

\newpage

%%%%%%%%%%%%%%%%%%%%%%
%%% Table of Contents
\tableofcontents\pdfbookmark[0]{Table of Contents}{toc}\newpage

The most used technique used to solve CSP puzzles like Sudoku is to first work out what is valid, In Sudoku the possible numbers that can go into a square can be worked out by checking the rows, columns and boxes (constraints), using this process to eliminate possible numbers can reduce the numbers to only one, this can then help solve other squares due to the new constraints, this is constraint propagation, numbers can also be decided by checking if a possible number can go anywhere else, if this is the only square this number can go in the row, column or box then it must go there. For easier puzzles this can find a solution or nearly complete solution alone. If two squares in the same box have the same possible numbers, then those numbers must go into one of those squares, meaning these numbers can be eliminated for other squares in the box, this is known as 'naked twins'. Sometimes these pairs are hidden by other possible numbers a square may be, if the numbers only go in that pair of squares, then the other numbers those squares could be can be eliminated from that square. This method can also be used to eliminate possible numbers from surrounding squares as you know that a number must go in this pair of squares, there are similar techniques to this that eliminate numbers by knowing that one number must be in a specific square.

There are other techniques that are not as useful for easier Sudokus, such as 'X-Wing', 'Swordfish' and 'forcing chain', as explained in ~\cite{KRISTANIX:2019}.

\label{endpage}

\newpage
\begin{thebibliography}{99}
	\addcontentsline{toc}{chapter}{Bibliography}
	\bibitem{KRISTANIX:2019}  Kristanix. \emph{kristanix.com/sudokuepic/sudoku-solving-techniques.php}.
\end{thebibliography}

\end{document}

%\end{article}
