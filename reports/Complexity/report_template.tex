\documentclass[]{final_report}
\usepackage{graphicx}
\usepackage{hyperref}


%%%%%%%%%%%%%%%%%%%%%%
%%% Input project details
\def\studentname{Luke Sell}
\def\reportyear{2019}
\def\projecttitle{Playing Games and Solving Puzzles Using AI}
\def\supervisorname{Iddo Tzameret}
\def\degree{BSc (Hons) in Computer Science}
\def\fullOrHalfUnit{Full Unit} % indicate if you are doing the project as a Full Unit or Half Unit
\def\finalOrInterim{Complexity and NP hardness} % indicate if this document is your Final Report or Interim Report

\begin{document}

\maketitle

%%%%%%%%%%%%%%%%%%%%%%
%%% Declaration

\chapter*{Declaration}

This report has been prepared on the basis of my own work. Where other published and unpublished source materials have been used, these have been acknowledged.

\vskip3em

Word Count: N/A

\vskip3em

Student Name: \studentname

\vskip3em

Date of Submission: 05/10/2019

\vskip3em

Signature: l.sell

\newpage

%%%%%%%%%%%%%%%%%%%%%%
%%% Table of Contents
\tableofcontents\pdfbookmark[0]{Table of Contents}{toc}\newpage

An n\textsuperscript{2} x n\textsuperscript{2} Sudoku is NP complete, as the problem is in NP, it can be represented as a decision problem, there either being a solution that can be verified in polynomial time or no solution and NP hard, as Sudoku can be represented as a graph colouring problem which is known to be NP hard as there is a polynomial time many-one reduction of this to Sudoku which implies it is NP hard.

\label{endpage}



\end{document}

%\end{article}
