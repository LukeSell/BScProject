\documentclass[]{final_report}
\usepackage{graphicx}
\usepackage{hyperref}


%%%%%%%%%%%%%%%%%%%%%%
%%% Input project details
\def\studentname{Luke Sell}
\def\reportyear{2019}
\def\projecttitle{Playing Games and Solving Puzzles Using AI}
\def\supervisorname{Iddo Tzameret}
\def\degree{BSc (Hons) in Computer Science}
\def\fullOrHalfUnit{Full Unit} % indicate if you are doing the project as a Full Unit or Half Unit
\def\finalOrInterim{Constraint Satisfaction} % indicate if this document is your Final Report or Interim Report

\begin{document}

\maketitle

%%%%%%%%%%%%%%%%%%%%%%
%%% Declaration

\chapter*{Declaration}

This report has been prepared on the basis of my own work. Where other published and unpublished source materials have been used, these have been acknowledged.

\vskip3em

Word Count: N/A

\vskip3em

Student Name: \studentname

\vskip3em

Date of Submission: 05/10/2019

\vskip3em

Signature: l.sell

\newpage

%%%%%%%%%%%%%%%%%%%%%%
%%% Table of Contents
\tableofcontents\pdfbookmark[0]{Table of Contents}{toc}\newpage

A Constraint Satisfaction Problem is modelled as having variables, domains of each variable and constraints between variables, Sudoku can be represented as such with each square being a variable, the domain being the numbers one to nine and the constraints being that a number cannot be repeated in a row, column or box. A variable can be made node consistent by eliminated all values in its domain that do not satisfy the variables unary constraints. A variable is arc consistent if all the values in its domain satisfy the variables binary constraints. This would mean that between two variables and for all values in the first variables domain there is a value in the second variables domain that satisfies the binary constraint on the arc joining them. This represents how constraint propagation can be modelled to reduce the possible numbers for all squares on a Sudoku grid and make solving the puzzle significantly easier.

\label{endpage}



\end{document}

%\end{article}
